\documentclass[master, och, assignment]{SCWorks}
% Тип обучения (одно из значений):
%    bachelor   - бакалавриат (по умолчанию)
%    spec       - специальность
%    master     - магистратура
% Форма обучения (одно из значений):
%    och        - очное (по умолчанию)
%    zaoch      - заочное
% Тип работы (одно из значений):
%    coursework - курсовая работа (по умолчанию)
%    referat    - реферат
%    otchet     - универсальный отчет
%    nirjournal - журнал НИР
%    diploma    - дипломная работа
%    pract      - отчет о научно-исследовательской работе
%    autoref    - автореферат выпускной работы
%    assignment - задание на выпускную квалификационную работу
%    review     - отзыв руководителя
%    critique   - рецензия на выпускную работу
% Включение шрифта
%    times      - включение шрифта Times New Roman (если установлен)
%                 по умолчанию выключен
\usepackage{preamble}

\usepackage{tempora}
\usepackage{cmap}


\begin{document}

% Кафедра (в родительном падеже)
\chair{информатики и программирования}

% Тема работы
\title{Разработка платформы единого резюме}

% Курс
\course{2}

% Группа
\group{273}

% Факультет (в родительном падеже) (по умолчанию "факультета КНиИТ")
% \department{факультета КНиИТ}

% Специальность/направление код - наименование
% \napravlenie{02.03.02 "--- Фундаментальная информатика и информационные технологии}
\napravlenie{02.04.03 "--- Математическое обеспечение и администрирование информационных систем}
% \napravlenie{09.03.01 "--- Информатика и вычислительная техника}
% \napravlenie{09.03.04 "--- Программная инженерия}
% \napravlenie{10.05.01 "--- Компьютерная безопасность}

% Для студентки. Для работы студента следующая команда не нужна.
% \studenttitle{Студентки}

% Фамилия, имя, отчество в родительном падеже
\author{Кулакова Максима Сергеевича}

% Руководитель НИР
\nirtitle{к.\,э.\,н., доцент} % степень, звание
\nirname{Л.\,В.\,Кабанова}

% Заведующий кафедрой
\chtitle{к.\,ф.-м.\,н.} % степень, звание
\chname{М.\,В.\,Огнева}

% Научный руководитель (для реферата преподаватель проверяющий работу)
\satitle{к.\,э.\,н., доцент} %должность, степень, звание
\saname{Л.\,В.\,Кабанова}

% Руководитель практики от организации (только для практики, для остальных типов
% работ не используется)
\patitle{к.\,э.\,н., доцент}
\paname{Л.\,В.\,Кабанова}

% Семестр (только для практики, для остальных типов работ не используется)
\term{1}

% Наименование практики (только для практики, для остальных типов работ не
% используется)
\practtype{производственная распределенная (научно-исследовательская работа)}

% Продолжительность практики (количество недель) (только для практики, для
% остальных типов работ не используется)
\duration{18}

% Даты начала и окончания практики (только для практики, для остальных типов
% работ не используется)
% \practStart{19.02.2024}
\practFinish{01.06.2024}

% Год выполнения отчета
\date{2024}

% \secrname{И.\,Ю.\,Мещерякова}
% \protnum{14}
% \protdate{19.02.2024}


\maketitle

% Включение нумерации рисунков, формул и таблиц по разделам (по умолчанию -
% нумерация сквозная) (допускается оба вида нумерации)
\secNumbering

% \tableofcontents

% Раздел "Обозначения и сокращения". Может отсутствовать в работе
% \abbreviations
% \begin{description}
%     \item ... "--- ...
%     \item ... "--- ...
% \end{description}

% Раздел "Определения". Может отсутствовать в работе
% \definitions

% Раздел "Определения, обозначения и сокращения". Может отсутствовать в работе.
% Если присутствует, то заменяет собой разделы "Обозначения и сокращения" и
% "Определения"
% \defabbr

% Ссылка на источник в тексте
% \cite{}

Постановка задачи: разработать платформу единого резюме, предоставляющую возможность автоматического обновления данных на различных сервисах поиска работы. 

Для решения данной задачи Кулакову М.С. необходимо:
\begin{enumerate}
    \item Рассмотреть и проанализировать существующие платформы для создания резюме с учетом их функционала, архитектуры и пользовательского опыта;
    \item Сформулировать собственные методы разработки единой платформы резюме, опираясь на литературный обзор и анализ существующих решений;
    \item Разработать и настроить клиентскую и серверную части платформы, выбрав оптимальные технические средства;
    \item Реализовать механизм взаимодействия платформы с различными сервисами поиска работы, обеспечивающий автоматическое обновление резюме на этих платформах.
\end{enumerate}

В теоретической части следует привести все необходимые сведения, необходимые для разработки и реализации платформы. В практической части следует описать реализацию платформы, включающей в себя агрегацию и обновление резюме на различных сервисах и продемонстрировать его работу. 

\signatureline

\end{document}
